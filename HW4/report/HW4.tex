\documentclass[a4paper,11pt]{article}
\usepackage[T1]{fontenc}
\usepackage[utf8]{inputenc}
\usepackage{lmodern}
\usepackage[english]{babel}
\usepackage{subcaption}
\usepackage{graphicx}
\usepackage{listings}
\usepackage{color}
\usepackage{hyperref}
\usepackage{titlesec}
\usepackage{amsmath}
\usepackage{tabularx}


\usepackage{siunitx}
\sisetup{output-exponent-marker=\ensuremath{\mathrm{e}}}

\titleformat{\section}{\normalfont\Large\bfseries}{Question \thesection}{1em}{}
% \renewcommand{\thesubsection}{\thesection.\alph{subsection}}

\definecolor{dkgreen}{rgb}{0,0.6,0}
\definecolor{gray}{rgb}{0.5,0.5,0.5}
\definecolor{mauve}{rgb}{0.58,0,0.82}

\lstset{frame=tb,
  language=Python,
  aboveskip=3mm,
  belowskip=3mm,
  showstringspaces=false,
%   columns=flexible,
  basicstyle={\small\ttfamily},
  numbers=left,
%   frame=single,
  morekeywords={mean,std},
  numberstyle=\tiny\color{gray},
  keywordstyle=\color{blue},
  commentstyle=\color{dkgreen},
  stringstyle=\color{mauve},
  breaklines=true,
  breakatwhitespace=true,
  tabsize=3
}

\title{Computer vision - Problem set 3}
\author{Nicolas Six}

\begin{document}

\maketitle

\section{Calibration}
\subsection{Computing M from ground truth}

\begin{lstlisting}
def gen_a(zipped_point):
    n = len(zipped_point)

    a = np.zeros(shape=(2 * n, 12), dtype=np.double)

    for i, ((p_x, p_y), (w_x, w_y, w_z)) in enumerate(zipped_point):
        a[2 * i, :] = [w_x, w_y, w_z, 1, 0, 0, 0, 0, -p_x * w_x, -p_x * w_y, -p_x * w_z, -p_x]
        a[2 * i + 1, :] = [0, 0, 0, 0, w_x, w_y, w_z, 1, -p_y * w_x, -p_y * w_y, -p_y * w_z, -p_y]

    return a


def compute_projection_matrix(zipped_point):
    a = gen_a(zipped_point)
    ata = np.matmul(a.T, a)
    val, vec = np.linalg.eig(ata)
    i = np.argmin(val)
    sol = vec[:, i]
    factor = 1 / np.sum(np.abs(np.power(sol, 2)))
    sol *= factor
    return np.reshape(sol, newshape=(3, 4))
\end{lstlisting}

\[
M=
\begin{bmatrix}
  -0.45827554 & 0.29474237 & 0.01395746 & -0.0040258\\
  0.05085589 & 0.0545847 & 0.54105993 & 0.05237592\\
  -0.10900958 & -0.17834548 & 0.04426782 & -0.5968205\\
\end{bmatrix}
\]

\[
\begin{bmatrix}
  u & v \\
\end{bmatrix}
 = 
\begin{bmatrix}
  0.14190608 & -0.45184301 \\
\end{bmatrix}
\]

\subsection{Compute M from randomly sampled point and ground truth}

\begin{lstlisting}
def compute_m_from_sample(full_point_list):
    min_diff = None
    best = None
    diff_list = []
    for s in [8, 12, 16]:
        for i in range(10):
            random.shuffle(full_point_list)
            m = compute_projection_matrix(full_point_list[:s])
            diff = 0.0
            for n in range(1, 5):
                p = full_point_list[-n]
                p3d = np.ones(shape=4)
                p3d[0:3] = p[1]
                p2d = np.ones(shape=3)
                p2d[0:2] = p[0]
                conv = np.matmul(m, p3d)
                conv /= conv[2]
                diff += np.sum(np.power(p2d - conv, 2))
            if min_diff is None or min_diff > diff:
                min_diff = diff
                best = (m, s, i)
            diff_list.append(diff / 4)
    return best, diff_list
\end{lstlisting}

\paragraph{}
As you can see on Table \ref{tab:residual}, the more point you take to build the estimator the lower is the maximum error. But at the same time minimum value as a tendency to rise, as you include more and more outliers to build our model. Therefor, most of the run we executed select a matrix from the $k=12$ column. This explain why the RANDSAC algorithm try a lot of combination of the smallest possible subset.


\newcolumntype{R}{>{\raggedright\arraybackslash}X}%
\newcolumntype{L}{>{\raggedleft\arraybackslash}X}%
\begin{table}
  \caption{Residual value get for ten run}
  \label{tab:residual}
  \begin{center}
    \begin{tabularx}{0.7\textwidth}{|r|L|L|L|}
     \hline
      \multicolumn{1}{|c}{Run} & \multicolumn{1}{|c}{$k=8$}  & \multicolumn{1}{|c}{$k=12$} & \multicolumn{1}{|c|}{$k=16$} \\ \hline
      1 & 3.081 & 1.873 & 2.090\\ 
      2 & 8.669 & 4.972 & 2.399\\ 
      3 & 3.331 & 2.340 & 0.949\\
      4 & 3.936 & 2.360 & 1.408\\ 
      5 & 1.803 & 1.183 & 1.003\\
      6 & 21.197 & 1.978 & 1.848\\ 
      7 & 1.978 & 5.080 & 2.762\\ 
      8 & 3.036 & 0.727 & 2.081\\ 
      9 & 0.858 & 0.597 & 4.547\\ 
      10 & 5.527 & 1.971 & 0.542\\
      \hline
    \end{tabularx}
  \end{center}
\end{table}

\[
M=
\begin{bmatrix}
  6.91763681e-03 & -3.99581895e-03 & -1.36628106e-03 & -8.27786171e-01\\
  1.54257321e-03 & 1.02297155e-03 & -7.25995610e-03 & -5.60924953e-01\\
  7.58555207e-06 & 3.71409261e-06 & -1.93437070e-06 & -3.38125837e-03\\
\end{bmatrix}
\]

\subsection{Camera world coordinate}

\begin{lstlisting}
c = - np.matmul(np.linalg.inv(m[0:3, 0:3]), m[:, 3])
\end{lstlisting}

\[
C=
\begin{bmatrix}
  303.10580556 & 307.17684589 & 30.42320286\\
\end{bmatrix}
\]

\section{Fundamental Matrix Estimation}

\subsection{Least square estimation of fundamental matrix}
\begin{lstlisting}
def gen_a(zipped_point):
    n = len(zipped_point)

    a = np.zeros(shape=(n, 9), dtype=np.double)

    for i, ((a_x, a_y), (b_x, b_y)) in enumerate(zipped_point):
        a[i, :] = [a_x * b_x, a_x * b_y, a_x,
                   a_y * b_x, a_y * b_y, a_y,
                   b_x,       b_y,       1]
    return a


def compute_projection_matrix(zipped_point):
    a = gen_a(zipped_point)
    ata = np.matmul(a.T, a)
    val, vec = np.linalg.eig(ata)
    i = np.argmin(val)
    sol = vec[:, i]
    factor = 1 / np.sum(np.abs(np.power(sol, 2)))
    sol *= factor
    return np.reshape(sol, newshape=(3, 3))
\end{lstlisting}

\[
\tilde{F} = 
\begin{bmatrix}
  -6.60698417e-07 & 7.91031620e-06 & -1.88600198e-03\\
  8.82396296e-06 & 1.21382933e-06 & 1.72332901e-02\\
  -9.07382303e-04 & -2.64234650e-02 & 9.99500092e-01\\
\end{bmatrix}
\]

\subsection{Rank reduced version of fundamental matrix}
\begin{lstlisting}
def reduce_rank(f):
    u, d, v = np.linalg.svd(f)
    d[2] = 0.0
    return np.mat(u) * np.mat(np.diag(d)) * np.mat(v)
\end{lstlisting}

\[
F =
\begin{bmatrix}
  -5.36264198e-07 & 7.90364770e-06 & -1.88600204e-03\\
  8.83539184e-06 & 1.21321685e-06 & 1.72332901e-02\\
  -9.07382265e-04 & -2.64234650e-02 & 9.99500092e-01\\
\end{bmatrix}
\]

\subsection{Drawing epipolar lines}
\begin{lstlisting}
def point_of_line(p, f):
    ph = np.mat([0, 0, 1])
    ph[0, :2] = p[:2]
    fx = np.mat(f) * ph.T
    b = 0
    p1 = ((b * fx[1] + fx[2]) / -fx[0], b)
    b = 712
    p2 = ((b * fx[1] + fx[2]) / -fx[0], b)
    return p1, p2


def draw_lines(img, src_pts, f):
    for pt in src_pts:
        p1, p2 = point_of_line(pt, f)
        p1 = tuple(map(int, p1))
        p2 = tuple(map(int, p2))
        cv2.line(img, p1, p2, (255, 0, 0), 1)
    return img
\end{lstlisting}

\begin{figure}[h!]
  \begin{center}
    \begin{subfigure}[t]{0.49\columnwidth}
      \centering
      \includegraphics[width=.99\linewidth]{{../Images/ps3-2-3-a}.png}
      \caption{Image A}
    \end{subfigure}
    \begin{subfigure}[t]{0.49\columnwidth}
      \centering
      \includegraphics[width=.99\linewidth]{{../Images/ps3-2-3-b}.png}
      \caption{Image B}
    \end{subfigure}
    \caption{Epipolar line computed from the point list provided, drawn on the original image}
    \label{q2}
  \end{center}
\end{figure}

\end{document}

